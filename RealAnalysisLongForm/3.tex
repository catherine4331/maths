\documentclass{article}
\usepackage{amsmath}
\usepackage{amssymb}
\usepackage{amsthm}
\setlength{\parindent}{0pt}

\author{Catherine Reid}
\title{Chapter 3 notes \& solutions}
\date{}

\begin{document}

\maketitle

\textbf{Theorem 3.5}: A sequence $(a_n)$ is bounded if and only if there exists some $C \in R$ such that $|a_n| \leq C$ for all $n$.

\begin{proof}
	Firstly assume that the sequence $(a_n)$ is bounded. Then there exists an upper and lower bound such that
	\begin{align*}
		L \leq a_n \leq U &  & \text{for all }n
	\end{align*}

	Let $C = \max(U, |L|)$. We note that this means $U \leq C$ and $-C \leq -|L|$, so we have
	\begin{align*}
		-C \leq -|L| \leq L \leq a_n \leq U \leq C
	\end{align*}

	And so
	\begin{align*}
		-C \leq a_n \leq C
	\end{align*}

	Therefore by proposition 1.12 $|a_n| \leq C$.

	Now, assume that there exists some $C \in R$ such that $|a_n| \leq C$ for all $n$.  By the defintion of the absolute value this means that
	\begin{align*}
		-C \leq a_n \leq C
	\end{align*}

	By setting $L = -C$ and $U = C$, we have shown that
	\begin{align*}
		L \leq a_n \leq U
	\end{align*}

	which means $(a_n)$ is bounded as required. This concludes the proof.
\end{proof}

\textbf{Example 3.9}: Show that the sequence
\begin{align*}
	(a_n) = \left(1, \frac{1}{2}, \frac{1}{3}, \frac{1}{4},...\right)
\end{align*}
converges to 0.

\begin{proof}
	Let $\varepsilon > 0$. Let $N = \tfrac{1}{\varepsilon}$. Then for any $n > N$ we have,
	\begin{align*}
		|a_n - a| = |\frac{1}{n} - 0| = \frac{1}{n} < \frac{1}{N} = \frac{1}{1/\varepsilon} = \varepsilon.
	\end{align*}

	That is $|a_n - a| < \varepsilon$. Therefore we have shown that $(a_n) \rightarrow 0$.
\end{proof}

\textbf{Example 3.10}: Let $a_n = 5 - \frac{1}{n^2}$. Show that $a_n \rightarrow 5$ as $n \rightarrow \infty$

\begin{proof}
	Let $\varepsilon > 0$. Let $N = \tfrac{1}{\sqrt{\varepsilon}}$. Then for any $n > N$ we have,
	\begin{align*}
		|a_n - a| = \left\lvert5 - \frac{1}{n^2} - 5\right\rvert = \frac{1}{n^2} < \frac{1}{N^2} = \frac{1}{1 / \sqrt{\varepsilon}^2} = \varepsilon.
	\end{align*}

	That is $|a_n - a| < \varepsilon$. Therefore we have shown that $(a_n) \rightarrow 5$.
\end{proof}

\textbf{Example 3.11}: Let $a_n = \dfrac{3n + 1}{n +2}$. Prove that $\lim_{x \to \infty} a_n = 3$

\begin{proof}
	Let $\varepsilon > 0$. Let $N = \tfrac{5}{\varepsilon} - 2$. Then for any $n > N$ we have,
	\begin{align*}
		|a_n - a| & = \left\lvert\frac{3n + 1}{n + 2} - 3\right\rvert                                   \\
		          & = \left\lvert\frac{3n + 1 - 3n - 6}{n + 2}\right\rvert                              \\
		          & = \frac{5}{n + 2} < \frac{5}{N + 2} = \frac{5}{5/\varepsilon - 2 + 2} = \varepsilon \\
	\end{align*}

	That is $|a_n - a| < \varepsilon$. Therefore we have shown that $(a_n) \rightarrow 3$.
\end{proof}

\textbf{Example 3.16}: Let $a_n = n^2$. Show that $\lim_{x \to \infty} a_n = \infty$
\begin{proof}
	Let $M > 0$. Let $N = \sqrt{M}$. Then for any $n > N$ we have,
	\begin{align*}
		a_n = n^2 > N^2 = \left(\sqrt{M}\right)^2 = M
	\end{align*}

	So we have shown that if $n > N$, then $a_n > M$. Thus $\lim_{x \to \infty} a_n = \infty$
\end{proof}

\textbf{Example 3.18}: Let $a_n = (-1)^n$. Prove that $(a_n)$ diverges

\begin{proof}
	Assume for contradiction that there is some $a$ such that $a_n \rightarrow a$. Let $\varepsilon = \tfrac{1}{2}$. Since we have assumed that
	$a_n \rightarrow a$, there must be some $N$ for which, for all $N > n$, we have $|a_n - a| < \frac{1}{2}$. That is $|(-1)^n - a| < \frac{1}{2}$ for all $n > N$.
	We proceed by cases.

	\begin{description}
		\item[Even $n$.]
		      If $n$ is even and $n > N$, then we have
		      \begin{align*}
			      \left\lvert1 - a\right\rvert & < \frac{1}{2}  \\
			      -\frac{1}{2} < 1 - a         & < \frac{1}{2}  \\
			      \frac{3}{2} < -a             & < -\frac{1}{2} \\
			      \frac {1}{2} < a             & < \frac{3}{2}
		      \end{align*}
		\item[Odd $n$.]
		      If $n$ is odd and $n > N$, then we have
		      \begin{align*}
			      \left\lvert-1 - a\right\rvert & < \frac{1}{2}  \\
			      \frac{1}{2} < -1 -a           & < \frac{1}{2}  \\
			      \frac{1}{2} < -a              & < \frac{3}{2}  \\
			      -\frac{3}{2} < a              & < -\frac{1}{2}
		      \end{align*}
	\end{description}

	But this is clearly a contradiction as no $a$ can be inside both of $(-\frac{3}{2}, -\frac{1}{2})$ and $(\frac{1}{2}, \frac{3}{2})$.
	Therefore $a_n \nrightarrow a$.
\end{proof}

We can use the triangle inequality for a more magical proof
\begin{proof}
	Again, let $\varepsilon = \frac{1}{2}$. The even case still gives $\left\lvert1 - a\right\rvert < \frac{1}{2}$ and the odd case
	gives $\left\lvert-1 - a\right\rvert < \frac{1}{2}$, although we will rewrite this as $\left\lvert1 + a\right\rvert < \frac{1}{2}$.
	Now, by the triangle inequality
	\begin{align*}
		2 = \left\lvert(1 + a) + (1 - a)\right\rvert \le |1 + a| + |1 - a|
	\end{align*}

	But we have assumed that the absolute values on the right are both less that $\frac{1}{2}$. So
	\begin{align*}
		2 = \left\lvert(1 + a) + (1 - a)\right\rvert \le |1 + a| + |1 - a| = \frac{1}{2} + \frac{1}{2} = 1
	\end{align*}

	This gives $2 < 1$, a contradiction
\end{proof}
\end{document}

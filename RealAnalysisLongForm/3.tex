\documentclass{article}
\usepackage{amsmath}
\usepackage{amssymb}
\usepackage{amsthm}
\setlength{\parindent}{0pt}

\author{Catherine Reid}
\title{Chapter 3 notes \& solutions}
\date{}

\begin{document}

\maketitle

\textbf{Theorem 3.5}: A sequence $(a_n)$ is bounded if and only if there exists some $C \in R$ such that $|a_n| \leq C$ for all $n$.

\begin{proof}
	Firstly assume that the sequence $(a_n)$ is bounded. Then there exists an upper and lower bound such that
	\begin{align*}
		L \leq a_n \leq U &  & \text{for all }n
	\end{align*}

	Let $C = \max(U, |L|)$. We note that this means $U \leq C$ and $-C \leq -|L|$, so we have
	\begin{align*}
		-C \leq -|L| \leq L \leq a_n \leq U \leq C
	\end{align*}

	And so
	\begin{align*}
		-C \leq a_n \leq C
	\end{align*}

	Therefore by proposition 1.12 $|a_n| \leq C$.

	Now, assume that there exists some $C \in R$ such that $|a_n| \leq C$ for all $n$.  By the defintion of the absolute value this means that
	\begin{align*}
		-C \leq a_n \leq C
	\end{align*}

	By setting $L = -C$ and $U = C$, we have shown that
	\begin{align*}
		L \leq a_n \leq U
	\end{align*}

	which means $(a_n)$ is bounded as required. This concludes the proof.
\end{proof}

\textbf{Example 3.9}: Show that the sequence
\begin{align*}
	(a_n) = \left(1, \frac{1}{2}, \frac{1}{3}, \frac{1}{4},...\right)
\end{align*}
converges to 0.

\begin{proof}
	Let $\varepsilon > 0$. Let $N = \tfrac{1}{\varepsilon}$. Then for any $n > N$ we have,
	\begin{align*}
		|a_n - a| = |\frac{1}{n} - 0| = \frac{1}{n} < \frac{1}{N} = \frac{1}{1/\varepsilon} = \varepsilon.
	\end{align*}

	That is $|a_n - a| < \varepsilon$. Therefore we have shown that $(a_n) \rightarrow 0$.
\end{proof}

\textbf{Example 3.10}: Let $a_n = 5 - \frac{1}{n^2}$. Show that $a_n \rightarrow 5$ as $n \rightarrow \infty$

\begin{proof}
	Let $\varepsilon > 0$. Let $N = \tfrac{1}{\sqrt{\varepsilon}}$. Then for any $n > N$ we have,
	\begin{align*}
		|a_n - a| = \left\lvert5 - \frac{1}{n^2} - 5\right\rvert = \frac{1}{n^2} < \frac{1}{N^2} = \frac{1}{1 / \sqrt{\varepsilon}^2} = \varepsilon.
	\end{align*}

	That is $|a_n - a| < \varepsilon$. Therefore we have shown that $(a_n) \rightarrow 5$.
\end{proof}

\textbf{Example 3.11}: Let $a_n = \dfrac{3n + 1}{n +2}$. Prove that $\lim_{x \to \infty} a_n = 3$

\begin{proof}
	Let $\varepsilon > 0$. Let $N = \tfrac{5}{\varepsilon} - 2$. Then for any $n > N$ we have,
	\begin{align*}
		|a_n - a| & = \left\lvert\frac{3n + 1}{n + 2} - 3\right\rvert                                   \\
		          & = \left\lvert\frac{3n + 1 - 3n - 6}{n + 2}\right\rvert                              \\
		          & = \frac{5}{n + 2} < \frac{5}{N + 2} = \frac{5}{5/\varepsilon - 2 + 2} = \varepsilon \\
	\end{align*}

	That is $|a_n - a| < \varepsilon$. Therefore we have shown that $(a_n) \rightarrow 3$.
\end{proof}
\end{document}

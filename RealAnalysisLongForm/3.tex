\documentclass{article}
\usepackage{amsmath}
\usepackage{amssymb}
\usepackage{amsthm}
\setlength{\parindent}{0pt}

\author{Catherine Reid}
\title{Chapter 3 notes \& solutions}
\date{}

\begin{document}

\maketitle

\textbf{Theorem 3.5}: A sequence $(a_n)$ is bounded if and only if there exists some $C \in R$ such that $|a_n| \leq C$ for all $n$.

\begin{proof}
	Firstly assume that the sequence $(a_n)$ is bounded. Then there exists an upper and lower bound such that
	\begin{align*}
		L \leq a_n \leq U &  & \text{for all }n
	\end{align*}

	Let $C = \max(U, |L|)$. We note that this means $U \leq C$ and $-C \leq -|L|$, so we have
	\begin{align*}
		-C \leq -|L| \leq L \leq a_n \leq U \leq C
	\end{align*}

	And so
	\begin{align*}
		-C \leq a_n \leq C
	\end{align*}

	Therefore by proposition 1.12 $|a_n| \leq C$.

	Now, assume that there exists some $C \in R$ such that $|a_n| \leq C$ for all $n$.  By the defintion of the absolute value this means that
	\begin{align*}
		-C \leq a_n \leq C
	\end{align*}

	By setting $L = -C$ and $U = C$, we have shown that
	\begin{align*}
		L \leq a_n \leq U
	\end{align*}

	which means $(a_n)$ is bounded as required. This concludes the proof.
\end{proof}

\textbf{Example 3.9}: Show that the sequence
\begin{align*}
	(a_n) = \left(1, \frac{1}{2}, \frac{1}{3}, \frac{1}{4},...\right)
\end{align*}
converges to 0.

\begin{proof}
	Let $\varepsilon > 0$. Let $N = \tfrac{1}{\varepsilon}$. Then for any $n > N$ we have,
	\begin{align*}
		|a_n - a| = |\frac{1}{n} - 0| = \frac{1}{n} < \frac{1}{N} = \frac{1}{1/\varepsilon} = \varepsilon.
	\end{align*}

	That is $|a_n - a| < \varepsilon$. Therefore we have shown that $(a_n) \rightarrow 0$.
\end{proof}

\textbf{Example 3.10}: Let $a_n = 5 - \frac{1}{n^2}$. Show that $a_n \rightarrow 5$ as $n \rightarrow \infty$

\begin{proof}
	Let $\varepsilon > 0$. Let $N = \tfrac{1}{\sqrt{\varepsilon}}$. Then for any $n > N$ we have,
	\begin{align*}
		|a_n - a| = \left\lvert5 - \frac{1}{n^2} - 5\right\rvert = \frac{1}{n^2} < \frac{1}{N^2} = \frac{1}{1 / \sqrt{\varepsilon}^2} = \varepsilon.
	\end{align*}

	That is $|a_n - a| < \varepsilon$. Therefore we have shown that $(a_n) \rightarrow 5$.
\end{proof}

\textbf{Example 3.11}: Let $a_n = \dfrac{3n + 1}{n +2}$. Prove that $\lim_{x \to \infty} a_n = 3$

\begin{proof}
	Let $\varepsilon > 0$. Let $N = \tfrac{5}{\varepsilon} - 2$. Then for any $n > N$ we have,
	\begin{align*}
		|a_n - a| & = \left\lvert\frac{3n + 1}{n + 2} - 3\right\rvert                                   \\
		          & = \left\lvert\frac{3n + 1 - 3n - 6}{n + 2}\right\rvert                              \\
		          & = \frac{5}{n + 2} < \frac{5}{N + 2} = \frac{5}{5/\varepsilon - 2 + 2} = \varepsilon \\
	\end{align*}

	That is $|a_n - a| < \varepsilon$. Therefore we have shown that $(a_n) \rightarrow 3$.
\end{proof}

\textbf{Example 3.16}: Let $a_n = n^2$. Show that $\lim_{x \to \infty} a_n = \infty$
\begin{proof}
	Let $M > 0$. Let $N = \sqrt{M}$. Then for any $n > N$ we have,
	\begin{align*}
		a_n = n^2 > N^2 = \left(\sqrt{M}\right)^2 = M
	\end{align*}

	So we have shown that if $n > N$, then $a_n > M$. Thus $\lim_{x \to \infty} a_n = \infty$
\end{proof}

\textbf{Example 3.18}: Let $a_n = (-1)^n$. Prove that $(a_n)$ diverges

\begin{proof}
	Assume for contradiction that there is some $a$ such that $a_n \rightarrow a$. Let $\varepsilon = \tfrac{1}{2}$. Since we have assumed that
	$a_n \rightarrow a$, there must be some $N$ for which, for all $N > n$, we have $|a_n - a| < \frac{1}{2}$. That is $|(-1)^n - a| < \frac{1}{2}$ for all $n > N$.
	We proceed by cases.

	\begin{description}
		\item[Even $n$.]
		      If $n$ is even and $n > N$, then we have
		      \begin{align*}
			      \left\lvert1 - a\right\rvert & < \frac{1}{2}  \\
			      -\frac{1}{2} < 1 - a         & < \frac{1}{2}  \\
			      \frac{3}{2} < -a             & < -\frac{1}{2} \\
			      \frac {1}{2} < a             & < \frac{3}{2}
		      \end{align*}
		\item[Odd $n$.]
		      If $n$ is odd and $n > N$, then we have
		      \begin{align*}
			      \left\lvert-1 - a\right\rvert & < \frac{1}{2}  \\
			      \frac{1}{2} < -1 -a           & < \frac{1}{2}  \\
			      \frac{1}{2} < -a              & < \frac{3}{2}  \\
			      -\frac{3}{2} < a              & < -\frac{1}{2}
		      \end{align*}
	\end{description}

	But this is clearly a contradiction as no $a$ can be inside both of $(-\frac{3}{2}, -\frac{1}{2})$ and $(\frac{1}{2}, \frac{3}{2})$.
	Therefore $a_n \nrightarrow a$.
\end{proof}

We can use the triangle inequality for a more magical proof
\begin{proof}
	Again, let $\varepsilon = \frac{1}{2}$. The even case still gives $\left\lvert1 - a\right\rvert < \frac{1}{2}$ and the odd case
	gives $\left\lvert-1 - a\right\rvert < \frac{1}{2}$, although we will rewrite this as $\left\lvert1 + a\right\rvert < \frac{1}{2}$.
	Now, by the triangle inequality
	\begin{align*}
		2 = \left\lvert(1 + a) + (1 - a)\right\rvert \le |1 + a| + |1 - a|
	\end{align*}

	But we have assumed that the absolute values on the right are both less that $\frac{1}{2}$. So
	\begin{align*}
		2 = \left\lvert(1 + a) + (1 - a)\right\rvert \le |1 + a| + |1 - a| = \frac{1}{2} + \frac{1}{2} = 1
	\end{align*}

	This gives $2 \le 1$, a contradiction
\end{proof}

\textbf{Proposition 3.19}: A sequence can not have more than one limit
\begin{proof}
	Suppose for contradiction that $a_n \rightarrow a$ and $a_n \rightarrow b$, where $a \neq b$. Without loss of generality,
	assume that $a < b$. Let $\varepsilon = (b - a)/ 3$. Since $a_n \rightarrow a$ there exists $N_1$ such that for $n > N_1$ we have
	$|a_n - a| < (b - a)/ 3$. Likewise since $a_n \rightarrow b$ there exists $N_2$ such that for $n > N_2$ we have $|a_n - b| < (b - a)/ 3$.
	Let $N = \max\{N_1, N_2\}$. Then for $n > N$, we have

	\begin{align*}
		-\frac{b - a}{3} < a_n - a < \frac{b - a}{3}   &  & \text{and} &  & -\frac{b - a}{3} < a_n - b < \frac{b - a}{3}    \\
		a -\frac{b - a}{3} < a_n < \frac{b - a}{3} + a &  & \text{and} &  & b - \frac{b - a}{3} < a_n < \frac{b - a}{3} + b
	\end{align*}

	In particular,
	\begin{align*}
		a_n < \frac{2a + b}{3} &  & \text{and} &  & \frac{2b + a}{3} < a_n
	\end{align*}

	Now, since $a < b$, we have $2a + b < 2b + a$. However this implies
	\begin{align*}
		a_n < \frac{2a + b}{3} < \frac{2b + a}{3} < a_n
	\end{align*}
	which is a contradiction.
\end{proof}

Again we can use the triangle inequality for a second proof
\begin{proof}
	Let $\varepsilon > 0$. Since $\varepsilon/2 > 0$ and $a_n \rightarrow a$, there exists some $N_1$ such that for $n > N_1$
	we have $|a_n - a| < \varepsilon/2$. Since $\varepsilon/2 > 0$ and $a_n \rightarrow b$, there exists some $N_2$ such that for $n > N_1$
	we have $|a_n - b| < \varepsilon/2$. Let $N = \max\{N_1, N_2\}$. Then for $n > N$, we have
	\begin{align*}
		|a - b| & = |a - a_n + a_n - b|           \\
		        & \le |a - a_n| + |a_n - b|       \\
		        & = |a_n - a| + |a_n - b|         \\
		        & < \varepsilon/2 + \varepsilon/2 \\
		        & = \varepsilon
	\end{align*}

	Since this holds for any $\varepsilon > 0$, we have shown that $|a - b| < \varepsilon$ for all $\varepsilon > 0$. This implies that
	$a = b$, showing that $a_n$ can only converge to a single limit.
\end{proof}

\textbf{Proposition 3.20}: If $(a_n)$ is a convergent sequence then $(a_n)$ is bounded.
\begin{proof}
	Assume that $(a_n) \rightarrow a$. Let $\varepsilon = 1$. Then there is some $N$ such that for $n > N$ we have
	\begin{align*}
		|a_n - a| < 1
	\end{align*}

	That is $-1 < a_n - a < 1$ for all $n > N$. Let
	\begin{align*}
		U = \max\{a_1, a_2, a_3, ..., a_N, a + 1\}
	\end{align*}
	and
	\begin{align*}
		L = \min\{a_1, a_2, a_3, ..., a_N, a - 1\}
	\end{align*}

	Note that if $n \le N$, then $L \le a_n \le U$, since each $a_n$ is included in the sets which we are taking the maximum
	and minimum of. And if $n > N$, then we have already noted that $a - 1 < a_n < a + 1$, which implies that
	\begin{align*}
		L \le a - 1 < a_n < a + 1 \le U,
	\end{align*}
	and hence $L \le a_n \le U$. Combining these cases gives
	\begin{align*}
		L \le a_n \le U
	\end{align*}
	for all $n$. Hence $(a_n)$ is bounded.
\end{proof}

\textbf{Theorem 3.23}: (Sequence squeeze theorem). Assume that $a_n \le x_n \le b_n$ for all $n$. Furthermore, assume that
\begin{align*}
	a_n \rightarrow L &  & \text{and} &  & b_n \rightarrow L
\end{align*}
Then,
\begin{align*}
	x_n \rightarrow L
\end{align*}

\begin{proof}
	Let $\varepsilon > 0$

	\begin{itemize}
		\item Since $(a_n) \rightarrow L$ then there is some $N_1$ such that for $n > N$ we have $|a_n - L| < \varepsilon$. That is,
		      $- \varepsilon < a_n - L < \varepsilon$. Or,
		      \begin{align*}
			      L - \varepsilon < a_n < L + \varepsilon
		      \end{align*}
		\item Since $(b_n) \rightarrow L$ then there is some $N_s$ such that for $n > N$ we have $|b_n - L| < \varepsilon$. That is,
		      $- \varepsilon < b_n - L < \varepsilon$. Or,
		      \begin{align*}
			      L - \varepsilon < b_n < L + \varepsilon
		      \end{align*}
	\end{itemize}

	Now, let $N = \max\{N_1, N_2\}$ and let $n > N$. Combining $a_n \le x_n \le b_n$ and the two inequalities above gives
	\begin{gather*}
		L - \varepsilon < a_n \le x_n \le b_n < L + \varepsilon \\
		L - \varepsilon       <         x_n < L + \varepsilon   \\
		-\varepsilon          <        x_n - L < \varepsilon    \\
		|x_n - L|             < \varepsilon
	\end{gather*}

\end{proof}
\end{document}
